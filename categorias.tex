% Created 2017-05-12 Fri 22:42
% Intended LaTeX compiler: pdflatex
\documentclass[11pt]{article}
\usepackage[utf8]{inputenc}
\usepackage[T1]{fontenc}
\usepackage{fixltx2e}
\usepackage{graphicx}
\usepackage{grffile}
\usepackage{longtable}
\usepackage{wrapfig}
\usepackage{rotating}
\usepackage[normalem]{ulem}
\usepackage{amsmath}
\usepackage{textcomp}
\usepackage{amssymb}
\usepackage{capt-of}
\usepackage[colorlinks=true]{hyperref}
\usepackage{tikz}
\usepackage{amsthm}
\usepackage{amsmath}
\usepackage{tikz-cd}
\theoremstyle{plain}
\newtheorem{theorem}{Teorema}
\newtheorem{fact}{Proposición}
\newtheorem{lemma}{Lema}
\newtheorem{corollary}{Corolario}
\theoremstyle{definition}
\newtheorem{definition}{Definición}
\newtheorem{proofs}{Demostración}
\theoremstyle{remark}
\newtheorem*{remark}{Nota}
\renewcommand*{\proofname}{Demostración}
\newenvironment{statement}{\noindent\rule[0.5ex]{\linewidth}{1pt}\begin{em}}{\end{em}\newline\noindent\rule[0.5ex]{\linewidth}{1pt}}
\DeclareMathOperator{\im}{Im}
\DeclareMathOperator{\coker}{Coker}
\DeclareMathOperator{\spec}{Spec}
\setlength{\parindent}{0pt}
\newcommand{\twopartdef}[4]{\left\{\begin{array}{ll} #1 & \mbox{if } #2 \\ #3 & \mbox{if } #4 \end{array} \right.}
\newcommand{\threepartdef}[6]{\left\{\begin{array}{lll} #1 & \mbox{if } #2 \\ #3 & \mbox{if } #4 \\ #5 & \mbox{if } #6 \end{array} \right.}
\usepackage{enumitem}
\setitemize{noitemsep,topsep=0pt,parsep=0pt,partopsep=0pt}
\setlist[enumerate]{topsep=0pt,itemsep=-1ex,partopsep=1ex,parsep=1ex}
\usepackage[top=0.9in, bottom=1.15in, left=1.15in, right=1.15in]{geometry}
\setlength\itemsep{0em}
\author{Mario Román}
\date{\today}
\title{Primera parte: Introducción a categorías}
\hypersetup{
 pdfauthor={Mario Román},
 pdftitle={Primera parte: Introducción a categorías},
 pdfkeywords={},
 pdfsubject={},
 pdfcreator={Emacs 26.0.50 (Org mode 9.0.5)}, 
 pdflang={Spanish}}
\begin{document}

\maketitle
\tableofcontents

\section{Categorías}
\label{sec:org36fe67c}
\subsection{Definición de categoría}
\label{sec:org84856c2}
\begin{definition}
Una \textbf{categoría} \({\cal C}\) está formada por \cite{brauer17} \cite{aluffi09_prelim}

\begin{itemize}
\item \({\cal C}_0\), un conjunto grande\footnote{: En la definición hemos usado explícitamente \emph{conjuntos grandes}, más
comúnmente llamados \emph{\href{https://es.wikipedia.org/wiki/Clase\_(teoría\_de\_conjuntos)}{clases}}, en lugar de conjuntos. En el futuro
querremos trabajar con categorías que contengan como objetos a todos
los conjuntos posibles; pero este tipo de construcciones causan problemas
en la teoría axiómatica de conjuntos tales como la 
\href{https://es.wikipedia.org/wiki/Paradoja\_de\_Russell}{paradoja de Russell}, así que usaremos esta definición más general para 
evitarlas.} cuyos elementos se llaman \textbf{objetos}, y
\item \({\cal C}_1\), un conjunto grande cuyos elementos se llaman \textbf{morfismos}.
\end{itemize}

A cada morfismo \(f \in {\cal C}_1\) se le asocian dos objetos, un \textbf{dominio}, \(\mathrm{dom}(f) \in {\cal C}_0\), 
y un \textbf{codominio} \(\mathrm{cod}(f) \in {\cal C}_0\); y solemos notarlo como

\[
f \colon \mathrm{dom}(f) \to \mathrm{cod}(f).
\]

Además, dados morfismos \(f \colon A \to B\) y \(g \colon B \to C\) existe su \textbf{morfismo composición}
\(g \circ f \colon A \to C\); la composición de morfismos se exige asociativa y con elementos
neutros \(\mathrm{id}_{A}\colon A \to A\). Es decir,

\[
h \circ (g \circ f) = (h \circ g) \circ f
\quad\text{ y }\quad
f \circ \mathrm{id}_A = f = \mathrm{id}_B \circ f.
\]
\end{definition}

\begin{remark}
Lo realmente importante en una categoría son los morfismos y cómo se
componen. Nótese que no trataremos el interior de los objetos; estos
sólo aparecen como dominio y codominio de los morfismos. Podemos
pensar en una categoría como la estructura algebraica que capta la
noción de composición.
\end{remark}

\begin{definition}
Notamos al conjunto de morfismos con dominio \(A\) y codominio \(B\) 
como \(\mathrm{Hom}(A,B)\).\footnote{: Nótese que la clase de morfismos entre dos objetos no
tiene por qué ser un conjunto en lugar de una clase. Las categorías con las
que trabajaremos, donde efectivamente es un conjunto, se llaman \textbf{categorías}
\textbf{localmente pequeñas}.}
\end{definition}

En ocasiones será útil señalar explícitamente la categoría sobre la que
se consideran los morfismos. En esos casos se notará con un subíndice,
como \(\mathrm{Hom}_{\cal C}(A,B)\). Al conjunto \(\mathrm{Hom}(A,A)\) lo llamamos conjunto de
\textbf{endomorfismos} del objeto \(A\) y lo notamos por \(\mathrm{End}(A)\).

\subsection{Ejemplos}
\label{sec:orga05288b}
\subsubsection{Ejemplos básicos}
\label{sec:orgba91929}
Vamos a crear una categoría como ejemplo. En nuestra categoría, los
objetos serán \(a,b,c\) y los morfismos serán \(\mathrm{id}_a,\mathrm{id}_b,\mathrm{id}_c,f,g,h\), con los 
dominios y codominios siguientes

\[\begin{tikzcd}
a  \ar{dr}[swap]{h} \rar{f} & 
b  \dar{g} \\
& c
\end{tikzcd}\]

donde la única composición posible, además de las de las identidades,
la definimos como \(f \circ g = h\). A esta forma de expresar morfismos la
llamamos \textbf{diagrama conmutativo}: escribimos una flecha entre dos
objetos para representar un morfismo entre ellos, las composiciones de
morfismos entre cualesquiera dos caminos entre dos objetos son iguales
y las identidades se omiten.

\begin{definition}
La \textbf{categoría vacía} se define como aquella con un conjunto de objetos
y de morfismos vacío.
\end{definition}

\begin{definition}
Una categoría de un solo objeto es un \textbf{monoide}.
\end{definition}

Para comprobar cómo coincide con nuestra definición usual de monoide,
debemos considerar los morfismos \(f \colon A \to A\) como elementos del monoide
y la composición como la operación del monoide. El elemento neutro será
la identidad del objeto.

\subsubsection{Categorías discretas}
\label{sec:orgb4d16c9}
\begin{definition}
Llamamos \textbf{categoría discreta} a aquella que sólo tiene identidades como
morfismos; no existen morfismos entre dos objetos distintos.
\end{definition}

Nótese que cada categoría discreta viene determinada por una clase de
objetos y cada clase (o conjunto) de objetos define una única categoría
discreta.

\subsubsection{Relaciones de orden}
\label{sec:org9fd3406}
\begin{lemma}
Cada conjunto parcialmente ordenado define una categoría con sus elementos
como objetos y un único morfismo \(\rho_{ab} \colon a \to b\) sólo cuando \(a \leq b\).
\end{lemma}

\begin{proof}
Definimos la composición de la única forma posible, \(\rho_{bc}\circ\rho_{ab}\) será \(\rho_{ac}\), el
único morfismo \(a \to c\), que además sabemos que existe porque si existen
\(\rho_{ab}\) y \(\rho_{bc}\), querrá decir que \(a \leq b\) y que \(b \leq c\); y por transitividad, \(a \leq c\).
La asociatividad se tiene trivialmente porque entre dos objetos habrá,
a lo sumo, un solo morfismo.

La identidad \(\mathrm{id}_a = \rho_{aa}\) existe por reflexividad y podemos comprobar 
trivialmente que es neutra respecto a la composición, ya que entre 
cualesquiera dos objetos existe a lo sumo un morfismo.
\end{proof}

Un ejemplo de relación de orden con sus morfismos podría ser la de
los divisores de \(12\) con la divisibilidad. En el siguiente diagrama
dibujamos algunos de sus morfismos omitiendo las identidades

\[\begin{tikzcd}
& 12 & \\
6 \urar & & 4 \ular \\
3 \ar{uur} \uar & & 2\uar\ar{ull}\ar{uul} \\
& 1 \ular\urar\ar{uuu}\ar{uur}\ar{uul} & &.
\end{tikzcd}\]

\begin{remark}
Cada \href{https://es.wikipedia.org/wiki/N\%25C3\%25BAmero\_ordinal\_(teor\%25C3\%25ADa\_de\_conjuntos)}{ordinal} define una relación de orden (un buen orden), que a su
vez define una categoría. Por ejemplo, la categoría asociada a \(1\)
será una categoría de un sólo objeto con la identidad, que también
puede verse como el monoide trivial. La categoría asociada a \(2\) será
de la forma

\[\begin{tikzcd}
1 \ar[loop above]{u}{\mathrm{id}} \rar{f} & 2 \ar[loop above]{u}{\mathrm{id}} &,
\end{tikzcd}\]

y la categoría asociada a \(\omega\) tendrá la forma

\[\begin{tikzcd}
1 \ar[loop above]{u} \rar \ar[bend right]{rr} \ar[bend right]{rrr} &
2 \ar[loop above]{u} \rar \ar[bend right]{rr} & 
3 \ar[loop above]{u} \rar \ar[bend right]{r} &
\dots & .
\end{tikzcd}\]
\end{remark}

\subsubsection{Conjuntos}
\label{sec:orgcb794da}
\begin{lemma}
Las aplicaciones entre conjuntos, con la composición usual, determinan
una categoría a la que llamaremos \(\mathtt{Set}\).
\end{lemma}

\begin{proof}
Sabemos que la composición usual de aplicaciones entre conjuntos es
asociativa y que la identidad es un elemento neutro con la composición.
\end{proof}

\subsubsection{Grupos abelianos, espacios vectoriales, módulos}
\label{sec:org072a0e2}
\begin{lemma}
Los homomorfismos de \(R\text{-módulos}\) izquierdos (respectivamente derechos)
con la composición usual, determinan una categoría a la que llamaremos
\(R\mathtt{-Mod}\) (respectivamente \(\mathtt{Mod-}R\)).
\end{lemma}

\begin{proof}
Nótese que de nuevo la composición usual es asociativa y la identidad
es elemento neutro de la composición. Sólo nos falta comprobar que,
efectivamente, la composición de homomorfismos de módulos sigue siendo
un homomorfismo de módulos.
\end{proof}

\begin{corollary}
Los homomorfismos de grupos abelianos forman una categoría que llamaremos
\(\mathtt{Ab}\). Las funciones lineales entre \(k\text{-espacios vectoriales}\) forman una categoría
a la que llamaremos \(k\mathtt{-Vect}\).
\end{corollary}

\begin{proof}
Nótese que los grupos abelianos son los \(\mathbb{Z}\text{-módulos}\) y que los
espacios vectoriales son los módulos sobre un cuerpo.
\end{proof}

\subsubsection{Espacios topológicos}
\label{sec:org1297640}
\begin{lemma}
Las funciones continuas entre espacios topológicos, con la composición
usual, determinan una categoría a la que llamaremos \(\mathtt{Top}\).
\end{lemma}

\begin{proof}
Sabemos que la composición usual de funciones continuas es
asociativa y que la identidad es un elemento neutro con la composición.
Falta comprobar que la composición de continuas es continua.
\end{proof}

\section{Morfismos}
\label{sec:orgf4eb595}
\subsection{Isomorfismos}
\label{sec:org3651552}
\begin{definition}
Un morfismo \(f\colon A \to B\) es un \textbf{isomorfismo} si existe un morfismo inverso
\(f^{-1} \colon B \to A\) cumpliendo que

\[
f \circ f^{-1} = \mathrm{id} = f^{-1} \circ f.
\]
\end{definition}

\begin{fact}
La inversa de un morfismo, si existe, es única.
\end{fact}
\begin{proof}
Supongamos un morfismo \(f\colon A \to B\) con dos inversas \(g_1,g_2\), se tiene entonces

\[g_1 = g_1\circ 1_B = g_1(fg_2) = 1_Ag_2 = g_2
.\] \cite{aluffi09_prelim}
\end{proof}

\begin{definition}
Dos objetos \(A,B\) son \textbf{isomorfos} si existe un isomorfismo entre
ellos. La isomorfía se nota como \(A \cong B\).
\end{definition}

\begin{fact}
Son isomorfismos

\begin{itemize}
\item la identidad, con \(\mathrm{id}^{-1} = \mathrm{id}\).
\item el inverso de un isomorfismo, con \(\left( f^{-1} \right)^{-1} = f\).
\item la composición de isomorfismos, con \(\left( g\circ f \right)^{-1} = f^{-1} \circ g^{-1}\).
\end{itemize}

La isomorfía es por tanto una relación de equivalencia entre objetos.
\end{fact}

\begin{definition}
Llamamos además \textbf{automorfismos} a los \emph{endomorfismos} que sean
\emph{isomorfismos}.
\end{definition}

\subsection{Monomorfismos y epimorfismos}
\label{sec:org5c609e9}
\begin{definition}
Un morfismo \(f\colon A \to B\) es un \textbf{monomorfismo} si puede cancelarse a la izquierda
en la composición. Es decir, para cualquier par de morfismos \(g,h\) se cumple

\[
f \circ g = f \circ h \implies g = h.
\]
\end{definition}

\begin{definition}
Un morfismo \(f\colon A \to B\) es un \textbf{epimorfismo} si puede cancelarse a la derecha
en la composición. Es decir, para cualquier par de morfismos \(g,h\) se cumple

\[
g \circ f = h \circ f \implies g = h.
\]
\end{definition}

\begin{remark}
Un morfismo puede ser monomorfismo y epimorfismo sin tener una inversa
y ser isomorfismo; en este caso lo llamamos \textbf{bimorfismo}.
\end{remark}

\subsection{Ejemplos}
\label{sec:orga8f9ae5}
\subsubsection{Aplicaciones inyectivas y sobreyectivas en conjuntos}
\label{sec:org09c6100}
\begin{proposition}
En la categoría de conjuntos,

\begin{itemize}
\item una aplicación es inyectiva si y sólo si es monomorfismo.
\item una aplicación es sobreyectiva si y sólo si es epimorfismo.
\item una aplicación es biyectiva si y sólo si es isomorfismo.
\end{itemize}
\end{proposition}

\begin{proof}
Si \(f \colon A \to B\) es inyectiva, es trivialmente monomorfismo. Si
no es inyectiva, existen \(x,y\) tales que \(f(x) \neq f(y)\). Podemos
tomar las funciones \(g,h \colon \left\{ \ast \right\} \to A\) definidas como \(g(\ast) = x\) y
como \(h(\ast) = y\) y comprobar que \(f\) no es monomorfismo.

Si \(f \colon A \to B\) es sobreyectiva, es trivialmente epimorfismo. Si
no es sobreyectiva, podemos tomar dos \(g,h \colon B \to B \cup \left\{ \ast,\ast' \right\}\) y que 
para un \(x \notin \mathrm{im}(f)\), se definan como \(g|_{B-\{x\}} = h|_{B-\{x\}} = \mathrm{id}\) pero con
\(g(x) = \ast\) y \(h(x) = \ast'\).
\end{proof}

\subsubsection{Grupos}
\label{sec:org3ad2a9b}
\begin{definition}
Una categoría de un sólo objeto en la que todos los morfismos son
isomorfismos es un \textbf{grupo}.
\end{definition}

Como en el caso de los monoides, podemos tomar los automorfismos del
objeto como los elementos del grupo.

\subsubsection{Un bimorfismo no isomorfismo}
\label{sec:orgd3bfff5}
Consideremos la inclusión \(i\colon \mathbb{Q} \to \mathbb{R}\) en \(\mathtt{Top}\). Trivialmente es un 
monomorfismo por ser inyectiva. Además, es epimorfismo, ya que si 
tenemos \(g,h \colon \mathbb{R} \to A\) cumpliendo \(g \circ i = h \circ
i\), tendríamos \(g|_{\mathbb{Q}} = h|_{\mathbb{Q}}\);
y por continuidad, tendríamos \(g = h\).

Sin embargo no es un isomorfismo, no existen homeomorfismos entre
\(\mathbb{Q}\) y \(\mathbb{R}\), que serían los isomorfismos de \(\mathtt{Top}\). Lo único que usamos es
que la densidad de \(\mathbb{Q}\) en \(\mathbb{R}\) fuerza a que sólo exista una posible
extensión continua de las funciones definidas en \(\mathbb{Q}\) a funciones
definidas en \(\mathbb{R}\).

\section{Construcciones universales}
\label{sec:orga91269d}
\subsection{Objetos terminales}
\label{sec:orge2e71c0}
\begin{definition}
Un objeto \(I\) se dice \textbf{inicial} si desde cualquier otro objeto llega exactamente
un morfismo. Es decir, para todo \(A\), existe un único \(i \colon I \to A\).
\end{definition}

\begin{definition}
Un objeto \(T\) se dice \textbf{final} si desde cualquier otro objeto sale exactamente
un morfismo. Es decir, para todo \(A\), existe un único \(t\colon A \to T\).
\end{definition}

\begin{definition}
Llamamos \textbf{objeto cero} a un objeto que es a la vez inicial y final.
\end{definition}

Nótese que una categoría no tiene por qué tener objetos terminales, ni tienen
por qué ser únicos; pero sí serán únicos salvo isomorfismos.

\begin{fact}
Los objetos iniciales y finales de una categoría son \emph{esencialmente únicos},
es decir,

\begin{itemize}
\item dos objetos iniciales serán isomorfos.
\item dos objetos finales serán isomorfos.
\end{itemize}
\end{fact}
\begin{proof}
Si existieran dos objetos iniciales \(a,b\), tendríamos un único morfismo dado
por \(f \colon a \to b\) y un único morfismo dado por \(g \colon b \to a\). Pero además, sólo existe
un endomorfismo de \(a\) y de \(b\), que es en ambos casos la identidad; así que
\(g \circ f \colon a \to a\) y \(f \circ g \colon b \to b\) son la identidad.
\end{proof}

\subsection{Productos y coproductos}
\label{sec:orgf0a964d}
\begin{definition}
Un objeto \(C\) es el \textbf{producto} de dos objetos \(A,B\) cuando existen morfismos

\[\begin{tikzcd}[rowsep=tiny]
& C \drar{\pi_B}\dlar[swap]{\pi_A} & \\
A && B
\end{tikzcd}\]

tales que para cualquier objeto \(D\) con un par de morfismos \(f_1\colon D \to A\) 
y \(f_2\colon D \to B\), existe un único \(h \colon D \to C\) tal que \(f_1 = \pi_A\circ h\) y
\(f_2 = \pi_B\circ h\), como muestra el siguiente diagrama


\[\begin{tikzcd}[rowsep=tiny]
& D \dar[dashed]{\exists! h} \ar[bend left]{ddr}{f_1}\ar[bend right,swap]{ddl}{f_2} & \\
& C \drar{\pi_B}\dlar[swap]{\pi_A} & \\
A && B &.
\end{tikzcd}\]
\end{definition}

\begin{remark}
El producto no tiene por qué existir.
\end{remark}

\begin{proposition}
El producto, si existe, es único.
\end{proposition}

\subsection{Ejemplos}
\label{sec:orgcc1e86e}
\subsubsection{Motivación del producto}
\label{sec:orgc69b4ef}
En la categoría de los conjuntos, podemos tomar los conjuntos
siguientes \(A = \left\{ \spadesuit, \clubsuit, \heartsuit, \diamondsuit \right\}\) y \(B = \left\{ 1,2,\dots,13  \right\}\).
preguntarnos si existe su producto. Un producto de ambos conjuntos 
tendrá que tener un morfismo hacia cada uno de ellos, así que 
estudiaremos varios candidatos y comprobaremos que sólo uno de
ellos cumple la condición del producto, poniendo de relieve en
el proceso la necesidad de esta condición.

\begin{itemize}
\item Un primer candidato a producto será el propio conjunto de símbolos
\(A\) con la identidad hacia
el propio \(A\) y una asignación constante de cualquier símbolo al
número \(1\), la función \(k(\ast) = 1\).

\[\begin{tikzcd}[rowsep=tiny]
   & A \dlar[swap]{\mathrm{id}}\drar{k} & \\
   A && B &.
   \end{tikzcd}\]

Sin embargo, podemos demostrar que este no es un producto válido.
En concreto, como la función \(k\) no es sobreyectiva, podemos
simplemente crear un par de morfismos desde el conjunto de un
sólo elemento, \(\left\{ x \right\}\), que lleguen a elementos fuera de su imagen

\[\begin{tikzcd}[rowsep=tiny]
   & \left\{ \ast \right\} \dar[dashed]{?} \ar[bend left]{ddr}{x \mapsto 3}\ar[bend right,swap]{ddl}{x \mapsto \diamondsuit} & \\
   & A \drar{k}\dlar[swap]{\mathrm{id}} & \\
   A && B &,
   \end{tikzcd}\]

y no podremos colocar ningún morfismo que haga cumplir la propiedad
del producto. De aquí extraemos que el producto debe ser
"suficientemente grande" como para que todos los morfismos puedan
escribirse pasando por él, \emph{cumplir la propiedad de existencia}.

\item Un segundo candidato a producto será \(D\), un conjunto formado por
dos números de \(B\) y un símbolo de \(A\), es decir

\[
   D = \left\{ n\ast m \mid \ast \in A, n,m \in B \right\}
   \]

\item Nuestro tercer candidato a producto será \(C\), el conjunto de los
pares de elementos de \(A\) y \(B\), es decir

\[
   C = \left\{ n\ast \mid \ast \in A, n \in B \right\},
   \]

que tendrá elementos como \(3\spadesuit\), con las proyecciones usuales \(p_\ast,p_n\),
que simplemente devuelven el número o el símbolo sin tener en cuenta el otro.

\[\begin{tikzcd}[rowsep=tiny]
   & C \dlar[swap]{p_1}\drar{p_2} & \\
   A && B &.
   \end{tikzcd}\]
\end{itemize}


\subsubsection{Categoría de los conjuntos}
\label{sec:org4f655ed}
En la categoría de los conjuntos, el conjunto vacío \(\varnothing\) es el objeto
inicial, mientras que cualquier conjunto de un elemento \(\left\{ \ast \right\}\) es el
objeto final. Nótese que el objeto final no es único, pero sí
\emph{esencialmente único}.

El producto de dos conjuntos será su producto cartesiano \(A \times B\) y
el coproducto de dos conjuntos será su unión disjunta \(A \sqcup B\).

\subsubsection{Categoría de las proposiciones}
\label{sec:org6058dd8}
Dado un \href{https://es.wikipedia.org/wiki/Dominio\_de\_discurso}{dominio de discurso}, una proposición lógica \(P\) implica otra
proposición lógica \(Q\) si en cualquier modelo en el que se verifique
\(P\) se verifica también \(Q\). Esto se nota como \(P \implies Q\). La
implicación es una relación transitiva y reflexiva
entre proposiciones; y por tanto, define una categoría que tiene como
objetos las proposiciones y un único morfismo entre cada dos
proposiciones sólo cuando una implica la otra.

El objeto inicial de esta categoría es una proposición que implica
todas las demás, la proposición falso \(F\), que es simplemente falsa
en cualquier modelo. La proposición falsa
\href{https://es.wikipedia.org/wiki/Principio\_de\_explosi\%25C3\%25B3n}{implica todas las demás} por reducción al absurdo. El objeto final
debe ser una proposición que es implicada por todas, la proposición
\(T\) que es verdadera en cualquier modelo.

El objeto producto de \(P\) y \(Q\) es \(P \wedge Q\), mientras que el objeto
coproducto de \(P\) y \(Q\) es \(P \vee Q\).

\section{Recursos}
\label{sec:org6abf146}
Vídeos y artículos de introducción a la teoría de categorías.

\begin{itemize}
\item Bartosz Milewski - \emph{\href{https://www.youtube.com/watch?v=I8LbkfSSR58\&list=PLbgaMIhjbmEnaH\_LTkxLI7FMa2HsnawM\_\&index=1}{Category theory for programmers (Youtube)}}.
\item Bartosz Milewski - \emph{\href{https://bartoszmilewski.com/category/category-theory/}{Posts on category theory}}.
\item The Catsters - \emph{\href{https://www.youtube.com/user/TheCatsters}{Videos on category theory}}.
\end{itemize}

Libros sobre álgebra y teoría de categorías básica.

\begin{itemize}
\item \textbf{Aluffi, Paolo - \emph{Algebra, Chapter 0}}. El inicio y la estructura de
estos apuntes están basados en este libro. Presenta una introducción
general al álgebra, pero usando como base la teoría de categorías.
\end{itemize}

\bibliographystyle{unsrt}
\bibliography{math}
\end{document}