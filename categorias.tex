\documentclass[a4paper, 11pt]{amsart}
\usepackage{amssymb}

%%% Castellano
\usepackage[spanish,es-noquoting]{babel} 
\selectlanguage{spanish}
\usepackage[utf8]{inputenc}

%%% Diagramas
% Conmutative diagrams
%\usepackage{tkz-graph}
%\usetikzlibrary{arrows}
\usepackage{tikz}
\usetikzlibrary{matrix,arrows}
\tikzset{every loop/.style={min distance=10mm,in=0,out=60,looseness=10}}

%% Multicolumns
\usepackage{multicol}

%%% Shortcuts
\newcommand{\C}{\mathcal{C} }

%%% Twopart definition
\newcommand{\twopartdef}[4]
{
	\left\{
		\begin{array}{ll}
			#1 & \mbox{si } #2 \\
			#3 & \mbox{si } #4
		\end{array}
	\right.
}



\newtheorem{theorem}{Teorema}[section]
\newtheorem{lemma}[theorem]{Lema}

\theoremstyle{definition}
\newtheorem{definition}[theorem]{Definición}
\newtheorem{example}[theorem]{Ejemplo}
\newtheorem{exca}[theorem]{Ejercicio}

\theoremstyle{remark}
\newtheorem{remark}[theorem]{Remark}

\numberwithin{equation}{section}

\begin{document}

\title{Introducción a la teoría de categorías}

%    Remove any unused author tags.

%    author one information
\author{Mario Román}
\address{}
\curraddr{}
\email{}
\thanks{}

% References
%\subjclass[2000]{Primary }
%    For articles to be published after 1 January 2010, you may use
%    the following version:
%\subjclass[2010]{Primary }

\keywords{}

\date{}

\dedicatory{}

\begin{abstract}
  Se explican tipos de morfismos y propiedades universales. Se hace especial incapié en productos y
  coproductos y se pasan a explicar los functores y las transformaciones naturales.
  Terminamos enunciando el Lema de Yoneda.
\end{abstract}

\maketitle

\section {Categorías}
  \subsection {Motivación}
    Varias estructuras matemáticas (grupos, espacios vectoriales, espacios topológicos \dots) cuentan
    con morfismos que preservan las estructura subyacentes entre ellas. Como ejemplos:
    \begin {center}
    \begin{tabular}{l|l}
      Conjunto & Morfismos \\
      \hline
      Grupos & Homomorfismos de grupos \\
      Espacios topológicos & Funciones continuas \\
      Espacios métricos & Funciones cortas \\
      Conjuntos & Funciones \\
      Espacios vectoriales sobre $\mathbb{K}$ & Funciones lineales sobre $\mathbb{K}$ \\
    \end{tabular}
    \end{center}
    Si estudiamos axiomáticamente las propiedades abstractas de estas estructuras y sus morfismos,
    obtendremos teoremas particularizables a todos estos casos útiles por sí mismos.
    Una categoría la formarán una clase de estos espacios con estructura y los morfismos entre estos
    espacios; y los teoremas que deduzcamos para todas las categorías podrán aplicarse a cada uno de
    los espacios.
    
  \subsection {Definición formal}
    \definition
    Una \textbf{categoría} $\C$ está definida por:
    \begin{itemize}
      \item Una clase de objetos de la categoría, $Obj(\mathcal{C})$.
      \item Un conjunto de morfismos $Hom_{\C}(A,B)$, poblado o no, entre cada par de objetos $A,B \in Obj(\C)$.
    \end{itemize}

    Cumpliendo sus morfismos las siguientes propiedades:
    \begin{itemize}
      \item Para dos morfismos $f \in Hom(A,B)$, $g \in Hom(B,C)$, existe su morfismo composición $f \circ g$.
      \item La composición es asociativa: $ f \circ (g \circ h) = (f \circ g) \circ h$
      \item Todos los objetos tienen un morfismo identidad, $1_{A} \in Hom(A,A)$, 
	  neutro para la composición: $\forall f \in Hom(A,B): f \circ 1_{A} = 1_{B} \circ f = f$
    \end{itemize}
    
    \exca Demostrar que la identidad es el único elemento neutro para la composición.
  
    \begin{center}
    	
    
    \begin{tabular}{ccc}
        \begin{tikzpicture}[descr/.style={fill=white,inner sep=2.5pt}]
	  \matrix (m) [matrix of math nodes, row sep=3em, column sep=3em]
	  { A & B \\
	     & C \\ };
	  \path[->,font=\scriptsize]
	  (m-1-1) edge node[auto] {$ f $} (m-1-2)
	  (m-1-2) edge node[auto] {$ g $} (m-2-2)
	  (m-1-1) edge node[auto,swap] {$ g \circ f $} (m-2-2);
	\end{tikzpicture} &
        
        \begin{tikzpicture}[descr/.style={fill=white,inner sep=2.5pt}]
	  \matrix (m) [matrix of math nodes, row sep=3em, column sep=3em]
	  { A & B & \\ & C & D \\ };
	  \path[->,font=\scriptsize]
	  (m-1-1) edge node[auto] {$ f $} (m-1-2)
	  (m-1-2) edge node[auto] {$ g $} (m-2-2)
	  (m-2-2) edge node[auto] {$ h $} (m-2-3)
	  (m-1-1) edge node[auto,swap] {$ g \circ f $} (m-2-2)
	  (m-1-2) edge node[auto] {$ h \circ g $} (m-2-3);
	\end{tikzpicture} &
        
	
        \begin{tikzpicture}[descr/.style={fill=white,inner sep=2.5pt}]
	  \matrix (m) [matrix of math nodes, row sep=3em, column sep=3em]
	  { A \\ };
	  \path[->,font=\scriptsize]
	  (m-1-1) edge[loop above] node[auto] {$ 1_A $} (m-1-1);
	\end{tikzpicture}
    \end{tabular}
    \end{center}
    \smallskip \textit{Diagramas conmutativos de las propiedades básicas.} \\
    
    
  \subsection {Ejemplos}
    Como idea simplificadora, podemos que los objetos son conjuntos, y los morfismos, funciones
    entre esos conjuntos; de hecho, el primer ejemplo es la definición de ese caso concreto. 
    Este es un buen modelo intuitivo para trabajar con algunas categorías,
    pero presentaremos ejemplos que rechazan esta intuición.
    \subsubsection{Categoría \texttt{Set}}
      La categoría de los conjuntos con las funciones entre conjuntos como morfismos.
      \begin{gather*}
        Obj(\texttt{Set}) = \{Todos\ los\ conjuntos\} \\
        Hom(A,B)= B^A = \{f \;|\; f: A \rightarrow B \}
      \end{gather*}
      Trivialmente es categoría: las funciones se componen en funciones, la composición es
      asociativa y la identidad funciona como se espera, siendo otra función.
      
    \subsubsection{Categoría \texttt{VectR}}
      La categoría de los espacios vectoriales reales con las funciones lineales entre
      espacios vectoriales reales.
      \begin{gather*}
        Obj(\texttt{VectR}) = \{Todos\ los\ espacios\ vectoriales\ sobre\ \mathbb{R}\} \\
        Hom(A,B)= \mathcal{L}_{\mathbb{R}}(A,B)
      \end{gather*}
      \exca Observar que \texttt{VectR} es una categoría.
    
    \subsubsection{Categoría \texttt{$(S,\sim)$}}
      Cualquier conjunto $S$ que tenga definida una relación de equivalencia $\sim$ tiene
      definida una categoría asociada en la que los objetos son los elementos del conjunto
      y los morfismos sólo representan casos particulares de la relación de equivalencia.
      \textit{En este ejemplo, los morfismos no son funciones y los objetos no son conjuntos,
      rechazando por primera vez la intuición del primer ejemplo.}
      \begin{gather*}
        Obj((S,\sim)) = S
      \end{gather*}
      Hay un morfismo entre dos elementos si y sólo si están relacionados:
      \begin{align*}
        Hom(a,b)= \twopartdef{(a,b)}{a \sim b}{\emptyset}{a \nsim b}
      \end{align*}
      Y la composición se define como:
      \begin{align*}
       (a,b) \circ (b,c) = (a,c)
      \end{align*}
      Probar que es categoría se reduce a notar que la composición de morfismos es morfismo (por
      ser la relación transitiva), que la composición es asociativa y que existe el morfismo identidad
      $(a,a)$, por ser la relación reflexiva.
     
    \subsubsection{Categoría \texttt{$(S,\leq)$}}
      Cualquier conjunto parcialmente ordenado tiene una categoría asociada. Los objetos
      son sus elementos y los morfismos son casos particulares de la relación de orden.
      \begin{gather*}
        Obj((S,\leq)) = S
      \end{gather*}
      Hay un morfismo entre $a$ y $b$ si y sólo si $a \leq b$:
      \begin{align*}
        Hom(a,b)= \twopartdef{(a,b)}{a \leq b}{\emptyset}{\mbox{no}}
      \end{align*}
      La composición se define como anteriormente: $(a,b) \circ (b,c) = (a,c)$, y es
      trivial volver a probar que se trata de una categoría.
      \medskip
      
      Un posible diagrama conmutativo de la categoría \texttt{$(\mathbb{N},\leq)$} sería:
      \begin{center}
        \begin{tikzpicture}[descr/.style={fill=white,inner sep=2.5pt}]
	  \matrix (m) [matrix of math nodes, row sep=3em, column sep=3em]
	  { 1 & 3 & 5 \\ & 6 & 7 \\ };
	  \path[->,font=\scriptsize]
	  (m-1-1) edge node[auto] {$ (1,3) $} (m-1-2)
	  (m-1-2) edge node[auto] {$ (3,6) $} (m-2-2)
	  (m-2-2) edge node[auto] {$ (6,7) $} (m-2-3)
	  (m-1-1) edge node[auto,swap] {$ (1,6) $} (m-2-2)
	  (m-1-2) edge node[auto] {$ (3,5) $} (m-1-3)
	  (m-1-3) edge node[auto] {$ (5,7) $} (m-2-3);
	\end{tikzpicture}
      \end{center}
    
\newpage
\section {Tipos de morfismos}
  \subsection {Isomorfismos}
    \definition Un morfismo $f \in Hom(A,B)$ es \textbf{isomorfismo} cuando existe un morfismo inverso
    $g \in Hom(B,A)$ cumpliendo:
    \begin{gather*}
     (g \circ f) = 1_A \qquad (f \circ g) = 1_B
    \end{gather*}
    \exca Probar que el inverso, si existe, es único. Lo notaremos como $f^{-1}$. Observar que si
    existe un inverso por la derecha y un inverso por la izquierda deben ser iguales.
    
    \subsubsection{Propiedades de isomorfismos}
      \begin{itemize}
       \item La identidad es isomorfismo: $(1)^{-1} = 1$ 
       \item El inverso de un isomorfismo es isomorfismo: $(f^{-1})^{-1} = f$.
       \item La composición de isomorfismos es isomorfismo: $(g \circ f)^{-1} = f^{-1} \circ g^{-1}$
      \end{itemize}

    Nótese que, precisamente, los isomorfismos de \texttt{Set} son las biyecciones entre conjuntos.
    Los isomorfismos de \texttt{Top} son los homeomorfismos y los isomorfismos de \texttt{Met} son
    las isometrías.
    
  \subsection {Epimorfismos y monomorfismos}

\newpage
\section {Propiedades universales}
  Varias construcciones en las diversas estructuras que motivaron las categorías pueden parecer
  escogidas arbitrariamente. Sin embargo, la definición de las propiedades universales las dejará
  como las únicas cumpliendo una construcción formal no arbitraria. Además, nos permitirá descubrir
  relaciones más profundas entre las construcciones en las distintas categorías.
  
  \subsection {Objetos terminales}
    \definition El objeto $I \in Obj(\C)$ se dice \textbf{inicial} cuando a cualquier otro objeto llega
    un único morfismo desde él. Es decir:
    \begin{gather*}
      \forall A \in Obj(\C):\quad \#(Hom(I,A)) = 1
    \end{gather*}

    \definition Análogamente, el objeto $F \in Obj(\C)$ se dice \textbf{final} cuando desde cualquier otro objeto llega
    un único morfismo hacia él. Es decir:
    \begin{gather*}
      \forall A \in Obj(\C):\quad \#(Hom(A,F)) = 1
    \end{gather*}
    
    Se llama \textbf{objeto terminal} a un objeto inicial o final y \textbf{objeto cero} a un objeto
    terminal y final. Una categoría no tiene por qué
    tener objetos terminales, y estos no tienen por qué ser únicos. Pero serán esencialmente únicos,
    es decir, si dos objetos son ambos iniciales o ambos finales en una categoría, serán isomorfos.
    
    \theorem Los objetos iniciales y los objetos finales son esencialmente únicos:
    \begin{itemize}
     \item Si $I_1,I_2$ son iniciales, $I_1 \cong I_2$
     \item Si $F_1,F_2$ son finales, $F_1 \cong F_2$
    \end{itemize}

    
  \subsection {Ejemplos}
    \subsubsection {Universales conjuntistas}
    \subsubsection {Proyección al cociente}
    
  \subsection {Propiedades universales de map, filter, fold}

\newpage
\section {Productos y coproductos}
  \subsection{Categorías $C_{A,B}$}
  \subsection{Producto y coproducto}
  \subsection{Casos particulares}
  
\newpage
\section {Functores}
  Una aplicación entre categorías, preservando su estructura, es un functor.
  \definition Sean $\mathcal{A}, \mathcal{B}$ categorías. Un \textbf{functor} 
  $F: \mathcal{A} \rightarrow \mathcal{B}$ está formado por:
  \begin{itemize}
   \item Una función entre objetos $F: Obj(\mathcal{A}) \rightarrow Obj(\mathcal{B})$
   \item Una función entre morfismos $F: Hom_\mathcal{A}(A,A') \rightarrow Hom_\mathcal{B}(F(A),F(A'))$
  \end{itemize}
  Cumpliendo:
  \begin{itemize}
   \item Respeta la composición: $F(f' \circ f) = F(f') \circ F(f)$
   \item Respeta la identidad: $F(1_A) = 1_{F(A)}$
  \end{itemize}

  
  Es decir, el functor lleva objetos de una categoría a objetos de otra y los morfismos
  entre objetos de una categoría a morfismos de entre las imágenes de los objetos en la
  otra categoría.

  \subsection {Functores de olvido}
    Un ejemplo de functores son los informalmente llamados functores de olvido,
    que \textit{olvidan} propiedades de una categoría al llevarla a otra mediante
    una inclusión.
    
    \example El functor $U: \mathtt{Grp} \rightarrow \mathtt{Set}$ que olvida la
    estructura de grupo. $U(G)$ es el conjunto de elementos del grupo $G$ y $U(\phi)$
    es la aplicación $\phi$ entre conjuntos, sin verla como un homomorfismo.
    
  \subsection {Functores libres}
    Los functores libres 
 
\newpage
\section {Transformaciones naturales}
  \definition Una transformación natural $\alpha$ entre dos functores $F,G: \C \rightarrow \mathcal{D}$
  viene determinada por:
  \begin{itemize}
   \item Para cada $X \in \C$, un morfismo: \, $\alpha_X : F(X) \rightarrow G(X)$
  \end{itemize}
  Cumpliendo que:
  \begin{itemize}
   \item Para cualquier morfismo $f \in Hom(X,Y)$ se tenga: \, $\alpha_Y \circ Ff = Gf \circ \alpha_X$
  \end{itemize}
  Lo que queda representado en el siguiente diagrama conmutativo de naturalidad:
  \begin{center}
  \begin{tikzpicture}[descr/.style={fill=white,inner sep=2.5pt}]
    \matrix (m) [matrix of math nodes, row sep=3em, column sep=3em]
    { FX & GX \\
      FY & GY \\ };
    \path[->,font=\scriptsize]
    (m-1-1) edge node[auto] {$ \alpha_X $} (m-1-2)
    (m-2-1) edge node[auto] {$ \alpha_Y $} (m-2-2)
    (m-1-2) edge node[auto] {$ Gf $} (m-2-2)
    (m-1-1) edge node[auto,swap] {$ Ff $} (m-2-1);
  \end{tikzpicture}
  \end{center}

\newpage
\section {Representables y el lema de Yoneda}

\end{document}

%-----------------------------------------------------------------------
% End of amsart.template
%-----------------------------------------------------------------------
