%%%
% Plantilla de Libro
% Modificación de una plantilla de Latex de Mathias Legrand (legrand.mathias@gmail.com)
% sobre modificaciones de Vel (vel@latextemplates.com) para adaptarla 
% al castellano y a las necesidades de escribir informática y matemáticas.
%
% Modificada por: Mario Román
%
% License:
% CC BY-NC-SA 3.0 (http://creativecommons.org/licenses/by-nc-sa/3.0/)
%%%

%%%%%%%%%%%%%%%%%%%%%%%%%%%%%%%%%%%%%%%%%
% The Legrand Orange Book
% LaTeX Template
% Version 2.0 (9/2/15)
%
% This template has been downloaded from:
% http://www.LaTeXTemplates.com
%
% Mathias Legrand (legrand.mathias@gmail.com) with modifications by:
% Vel (vel@latextemplates.com)
%
% License:
% CC BY-NC-SA 3.0 (http://creativecommons.org/licenses/by-nc-sa/3.0/)
%
% Compiling this template:
% This template uses biber for its bibliography and makeindex for its index.
% When you first open the template, compile it from the command line with the 
% commands below to make sure your LaTeX distribution is configured correctly:
%
% 1) pdflatex main
% 2) makeindex main.idx -s StyleInd.ist
% 3) biber main
% 4) pdflatex main x 2
%
% After this, when you wish to update the bibliography/index use the appropriate
% command above and make sure to compile with pdflatex several times 
% afterwards to propagate your changes to the document.
%
% This template also uses a number of packages which may need to be
% updated to the newest versions for the template to compile. It is strongly
% recommended you update your LaTeX distribution if you have any
% compilation errors.
%
% Important note:
% Chapter heading images should have a 2:1 width:height ratio,
% e.g. 920px width and 460px height.
%
%%%%%%%%%%%%%%%%%%%%%%%%%%%%%%%%%%%%%%%%%

%----------------------------------------------------------------------------------------
%	PACKAGES AND OTHER DOCUMENT CONFIGURATIONS
%----------------------------------------------------------------------------------------

%%% Configuración del papel.
% fleqn: Alinea ecuaciones a la izquierda
\documentclass[11pt, fleqn, spanish]{book}

%%% Castellano.
% noquoting: Permite uso de comillas no españolas.
% lcroman: Permite la enumeración con numerales romanos en minúscula.
% fontenc: Usa la fuente completa para que pueda copiarse correctamente del pdf.
\usepackage[spanish,es-noquoting,es-lcroman]{babel}
\usepackage[utf8]{inputenc}
\usepackage[T1]{fontenc}
\selectlanguage{spanish}

%%% Matemáticas.
% Paquetes de la AMS. Para entornos de ecuaciones.
\usepackage{amsmath,amsfonts,amsthm}

% Paquete propio para escribir teoría de categorías
\usepackage{categoriesdiag}


% Conmutative diagrams
%\usepackage{tkz-graph}
%\usetikzlibrary{arrows}
\usepackage{tikz}
\usetikzlibrary{matrix,arrows}
\tikzset{every loop/.style={min distance=10mm,in=0,out=60,looseness=10}}
\usepackage{tikz-cd}
\usetikzlibrary{cd}

%% Multicolumns
\usepackage{multicol}

%%% Shortcuts
\newcommand{\C}{\mathcal{C}}

%%% Twopart definition
\newcommand{\twopartdef}[4]
{
  \left\{
    \begin{array}{ll}
      #1 & \mbox{si } #2 \\
      #3 & \mbox{si } #4
    \end{array}
  \right.
}


%----------------------------------------------------------------------------------------

\input{structure} % Insert the commands.tex file which contains the majority of the structure behind the template

\begin{document}

%----------------------------------------------------------------------------------------
%	TÍTULO
%----------------------------------------------------------------------------------------

\begingroup
\thispagestyle{empty}
\begin{tikzpicture}[remember picture,overlay]
\coordinate [below=12cm] (midpoint) at (current page.north);
\node at (current page.north west)
{\begin{tikzpicture}[remember picture,overlay]
\node[anchor=north west,inner sep=0pt] at (0,0) {\includegraphics[width=\paperwidth]{background}}; % Background image
\draw[anchor=north] (midpoint) node [fill=ocre!30!white,fill opacity=0.6,text opacity=1,inner sep=1cm]{\Huge\centering\bfseries\sffamily\parbox[c][][t]{\paperwidth}
{\centering Teoría de Categorías\\[15pt] % Book title
{\Large Introducción}\\[20pt] % Subtitle
{\Large Mario Román - LibreIM}}}; % Author name
\end{tikzpicture}};
\end{tikzpicture}
\vfill
\endgroup

%----------------------------------------------------------------------------------------
%	COPYRIGHT
%----------------------------------------------------------------------------------------

\newpage
~\vfill
\thispagestyle{empty}

\noindent Copyright \copyright\ 2017 Mario Román\\ % Copyright notice
\noindent \textsc{LibreIM}\\ % Publisher
\noindent \textsc{tux.ugr.es/libreim}\\ % URL

\noindent Licensed under the Creative Commons Attribution-NonCommercial 3.0 Unported License 
(the ``License''). You may not use this file except in compliance with the License. You may 
obtain a copy of the License at \url{http://creativecommons.org/licenses/by-nc/3.0}. Unless 
required by applicable law or agreed to in writing, software distributed under the License 
is distributed on an \textsc{``as is'' basis, without warranties or conditions of any kind}, 
either express or implied. See the License for the specific language governing permissions 
and limitations under the License.\\ % License information

%\noindent \textit{First printing, March 2015} % Printing/edition date

%----------------------------------------------------------------------------------------
%	CONTENIDOS
%----------------------------------------------------------------------------------------

\chapterimage{chapter_head_1.pdf} % Table of contents heading image
\pagestyle{empty} % No headers
\tableofcontents % Print the table of contents itself
\cleardoublepage % Forces the first chapter to start on an odd page so it's on the right
\pagestyle{fancy} % Print headers again

%----------------------------------------------------------------------------------------
%	PART
%----------------------------------------------------------------------------------------

\part{Teoría de Categorías}

%----------------------------------------------------------------------------------------
%	CHAPTER 1
%----------------------------------------------------------------------------------------

\chapterimage{chapter_head_2.pdf} % Chapter heading image

\chapter{Categorías}

\section{Motivación}
\index{Motivación}

\subsection{Estructuras matemáticas}
Varias estructuras matemáticas (grupos, espacios vectoriales, espacios
topológicos \dots) cuentan con morfismos que preservan las estructuras
subyacentes entre ellas. Como ejemplos: \citation{book_key} 

\begin{center}
  \begin{tabular}{l|l}
    Estructuras & Morfismos \\
    \hline
    Grupos & Homomorfismos de grupos \\
    Espacios topológicos & Funciones continuas \\
    Espacios métricos & Funciones cortas \\
    Conjuntos & Funciones \\
    Espacios vectoriales sobre $\mathbb{K}$ & Funciones lineales sobre $\mathbb{K}$ \\
  \end{tabular}
\end{center}

Si estudiamos axiomáticamente las propiedades abstractas de estas estructuras y sus morfismos,
obtendremos teoremas particularizables a todos estos casos, útiles por sí mismos.
Una categoría la formarán una clase de estos espacios con estructura y los morfismos entre estos
espacios; y los teoremas que deduzcamos para todas las categorías podrán aplicarse a cada uno de
los espacios.

\subsection{Composición}
La categoría es una estructura algebraica que modela, de alguna forma,
lo que puede ser componible. Las funciones entre conjuntos son un ejemplo
básico pero no son el único: los programas de un lenguaje tipado se componen
entre sí o las relaciones de orden (gracias a la transitividad) se componen
entre sí.

\begin{center}
  \begin{tabular}{l|l}
    Objetos & Morfismos \\
    \hline
    Tipos de datos & Funciones puras \\
    Elementos ordenados & Relación de orden \\
    Proposiciones & Implicaciones
  \end{tabular}
\end{center}


\section {Definición formal}
% Definición de categoría
\begin{definition}
  Una \textbf{categoría} $\C$ está definida por:
  
  \begin{itemize}
  \item Una clase de objetos de la categoría, $Obj(\mathcal{C})$.
  \item Un conjunto de morfismos $Hom_{\C}(A,B)$, poblado o no, entre
    cada par de objetos $A,B \in Obj(\C)$.
  \end{itemize}
  
  Cumpliendo sus morfismos las siguientes \textbf{propiedades de la composición}:

  \begin{itemize}
  \item Para dos morfismos $f \in Hom(A,B)$, $g \in Hom(B,C)$, existe su
    morfismo composición $f \circ g$.
  \item La composición es asociativa: $f \circ (g \circ h) = (f \circ g) \circ h$
  \item Todos los objetos tienen un morfismo identidad,
    $1_{A} \in Hom(A,A)$, neutro para la composición:
    $\forall f \in Hom(A,B): f \circ 1_{A} = 1_{B} \circ f = f$
  \end{itemize}
\end{definition}

Es decir, para definir una categoría tenemos por un lado que definir
sus objetos, que son las entidades entre las que actuarán los
morfismos, y, por otro lado, los morfismos entre cada par de objetos.
Nótese que en el ejemplo de los grupos, los conjuntos o los espacios
topológicos, los objetos son conjuntos y los morfismos funciones entre
ellas. La definición de categoría es independiente de los elementos
internos de los conjuntos. \\

Si quitamos importancia a los objetos, la categoría es simplemente un
conjunto de morfismos que pueden componerse sólo cuando el
\textit{dominio} de uno coincida con el \textit{codominio} del otro.

\begin{exercise}
  Demostrar que la identidad es el único elemento neutro para la
  composición.
\end{exercise}

\begin{center}    
  \begin{tabular}{ccc}
    \begin{tikzpicture}[descr/.style={fill=white,inner sep=2.5pt}]
      \matrix (m) [matrix of math nodes, row sep=3em, column sep=3em]
      { A & B \\
        & C \\ };
      \path[->,font=\scriptsize]
      (m-1-1) edge node[auto] {$ f $} (m-1-2)
      (m-1-2) edge node[auto] {$ g $} (m-2-2)
      (m-1-1) edge node[auto,swap] {$ g \circ f $} (m-2-2);
    \end{tikzpicture} & \begin{tikzpicture}[descr/.style={fill=white,inner sep=2.5pt}]
      \matrix (m) [matrix of math nodes, row sep=3em, column sep=3em]
      { A & B & \\ & C & D \\ };
      \path[->,font=\scriptsize]
      (m-1-1) edge node[auto] {$ f $} (m-1-2)
      (m-1-2) edge node[auto] {$ g $} (m-2-2)
      (m-2-2) edge node[auto] {$ h $} (m-2-3)
      (m-1-1) edge node[auto,swap] {$ g \circ f $} (m-2-2)
      (m-1-2) edge node[auto] {$ h \circ g $} (m-2-3);
    \end{tikzpicture} & \begin{tikzpicture}[descr/.style={fill=white,inner sep=2.5pt}]
      \matrix (m) [matrix of math nodes, row sep=3em, column sep=3em]
      { A \\ };
      \path[->,font=\scriptsize]
      (m-1-1) edge[loop above] node[auto] {$ 1_A $} (m-1-1);
    \end{tikzpicture}
  \end{tabular}
\end{center}
\smallskip 
\textit{Diagramas conmutativos de las propiedades básicas.} \\

Usaremos \textbf{diagramas conmutativos} para representar partes de una categoría.
Cada morfismo lo representamos por una flecha de su codominio a su dominio;
y asumimos que cualesquiera dos caminos en el diagrama dan lugar al mismo
morfismo compuesto.

\section{Ejemplos de categorías}
Como idea simplificadora, podríamos pensar que los objetos son
conjuntos, y los morfismos, funciones entre esos conjuntos; de hecho,
el primer ejemplo es ese caso concreto. Este es un buen modelo
intuitivo para trabajar con algunas categorías, pero presentaremos
ejemplos que rechazan esta intuición. Hay categorías que no consisten
de conjuntos y morfismos entre ellos.

\subsubsection{Categoría de un objeto}
Podemos crear una categoría con un sólo objeto $A$ y definir sobre él
morfismos que tendrán que pertenecer forzosamente a $Hom(A,A)$. Nótese
como entre estos morfismos, deberá existir una identidad y una composición
asociativa. Así, las categorías de un sólo objeto son los \textbf{monoides}.

\subsubsection{Una categoría de dos objetos}
Podemos crear categorías pequeñas tomando varios objetos, definiendo unos pocos
morfismos entre ellos y asegurándonos que cada par de morfismos tiene su composición
en la categoría:

\begin{tikzcd}
  A \lar[loop right]{\mathrm{id}} \rar[bend left]{f}\rar[bend right]{g}
  &
  B \arrow[loop right]{r}{\mathrm{id}}
\end{tikzcd}

\subsubsection{Categoría \texttt{Set}}
La categoría de los conjuntos con las funciones entre conjuntos como morfismos.

\begin{gather*}
  Obj(\texttt{Set}) = Clase\ de\ todos\ los\ conjuntos \\
  Hom(A,B)= B^A = \{f \;|\; f: A \rightarrow B \}
\end{gather*}

La clase de todos los conjuntos forma en sí misma una categoría, que
tiene como objetos a los conjuntos, y como morfismos a las funciones
entre conjuntos con la composición usual. Podemos comprobar que las
funciones cumplen las propiedades de la composición. \\

Hay además un detalle que tenemos que tener en cuenta. Estamos hablando
de la \textit{clase de todos los conjuntos} porque no podemos hablar
propiamente del \textit{conjunto de todos los conjuntos}; definirlo nos
llevaría a tener que tratar con problemas como la \href{https://es.wikipedia.org/wiki/Paradoja_de_Russell}{paradoja de Russell}.

\begin{tikzcd}
                                        & \{1,2\} \dar{f} \\
\varnothing \urar[hook]{i}\rar[hook]{i} & \{a,b,c\}
\end{tikzcd}

\textit{Diagrama conmutativo en la categoría de los conjuntos.} \\


\subsubsection{Categoría \texttt{k-Vect}}
La categoría de los espacios vectoriales sobre un cuerpo fijo $k$, con
las funciones lineales entre espacios vectoriales reales.

\begin{gather*}
  Obj(\texttt{VectR}) = Clase\ de\ espacios\ vectoriales\ sobre\ \mathbb{R} \\
  Hom(A,B)= \mathcal{L}_{\mathbb{R}}(A,B)
\end{gather*}

Donde cada función $f \in {\cal L}_k(A,B)$ cumple que $f(x+y) = f(x)+f(y)$ para cualesquiera $x,y \in A$;
y además $f(\alpha x) = \alpha f(x)$ cuando $\alpha \in k$.

\begin{exercise} 
  Observar que \texttt{VectR} es una categoría. Debe comprobarse que
  la composición de funciones lineales es lineal y que la identidad sea
  también lineal.
\end{exercise}

Nótese que cada cuerpo $k$ da lugar a una categoría distinta de espacios
vectoriales sobre él.


\subsubsection{Categoría \texttt{Grp}}
La categoría de los grupos, con los homomorfismos de grupos entre ellos.

\begin{gather*}
  Obj(\texttt{Grp}) = Grupos \\
  Hom(A,B)= Hom_{grp}(A,B)
\end{gather*}

Donde un $f \in Hom_{grp}(A,B)$ debe cumplir $f(xy) = f(x)f(y)$ para cualesquiera $x,y \in A$.

\subsubsection{Categoría de espacios topológicos}
La categoría de los espacios topológicos, con las funciones continuas entre ellos.

\begin{gather*}
  Obj(\texttt{Grp}) = Espacios\ topológicos \\
  Hom(A,B)= Hom_{top}(A,B)
\end{gather*}

Donde a un $f \in Hom_{top}(A,B)$ le exigimos que sea continuo. Esto es, que dado
un abierto $U \in B$, su preimagen, $f^{-1}(U)$, sea abierta en $A$.

\begin{exercise}
  Observar que \texttt{Top} es una categoría.
\end{exercise}


\subsubsection{Categoría \texttt{$(S,\sim)$}}
Cualquier conjunto $S$ que tenga definida una relación de equivalencia $\sim$ tiene
definida una categoría asociada en la que los objetos son los elementos del conjunto
y los morfismos sólo representan casos particulares de la relación de equivalencia.
\textit{En este ejemplo, los morfismos no son funciones y los objetos no son conjuntos,
  rechazando por primera vez la intuición del primer ejemplo.}

\begin{gather*}
  Obj((S,\sim)) = S
\end{gather*}

Hay un morfismo entre dos elementos si y sólo si están relacionados:

\begin{align*}
  Hom(a,b)= \twopartdef{(a,b)}{a \sim b}{\emptyset}{a \nsim b}
\end{align*}





\chapter{Tipos de morfismos}

\section {Isomorfismos}
Queremos dar una definición que sea equivalente al habitual
isomorfismo entre conjuntos pero que provenga de la estructura de
categoría. Así, podremos llevar la noción de isomorfismo a categorías
más allá de \texttt{Set}.

\begin{definition}
  Un morfismo $f \in Hom(A,B)$ es \textbf{isomorfismo} si existe un morfismo inverso:
  $g \in Hom(B,A)$ cumpliendo:
  \begin{gather*}
    (g \circ f) = 1_A \qquad (f \circ g) = 1_B
  \end{gather*}
\end{definition}
  
\begin{exercise} 
  Probar que el inverso, si existe, es único. Lo notaremos como $f^{-1}$. Observar que si
  existe un inverso por la derecha y un inverso por la izquierda, deben ser iguales.
\end{exercise}

Nótese que no hemos probado todavía (aunque veremos que es trivial),
que la noción de isomorfismo que acabamos de definir coincida con la
noción habitual de isomorfismo entre conjuntos.


\subsection{Propiedades de isomorfismos}
\begin{theorem}
  Los isomorfismos de cualquier categoría cumplen que:
  
  \begin{itemize}
  \item La identidad es isomorfismo: $(1)^{-1} = 1$ 
  \item El inverso de un isomorfismo es isomorfismo: $(f^{-1})^{-1} = f$.
  \item La composición de isomorfismos es isomorfismo: $(g \circ f)^{-1} = f^{-1} \circ g^{-1}$
  \end{itemize}

\end{theorem}

Nótese que, precisamente, los isomorfismos de \texttt{Set} son las
biyecciones entre conjuntos.  Los isomorfismos de \texttt{Top} son los
homeomorfismos y los isomorfismos de \texttt{Met} son las isometrías.

\begin{definition}
  Dos objetos $A,B$ son \textbf{isomorfos} si existe un isomorfismo en
  $Hom(A,B)$, la isomorfía se nota como $A \cong B$ y es, por las
  propiedades anteriores, una relación de equivalencia.
\end{definition}


\section{Epimorfismos y monomorfismos}
Una vez hemos llevado la noción de isomorfismo a cualquier categoría, parece
razonable definir nociones equivalentes para las funciones que sólo presentan
inversa por un lado.

\begin{definition} 
  Un morfismo $f \in Hom(A,B)$ es \textbf{monomorfismo} si para
  cualquier objeto $C$ y para cada par de morfismos
  $g,h: C \rightarrow A$, el morfismo puede \textit{cancelarse} a la
  izquierda:
  
  \begin{gather*}
    f \circ g = f \circ h \quad \Rightarrow \quad g = h
  \end{gather*}
\end{definition}

Es decir, si el siguiente diagrama conmuta y $f$ es monomorfismo,
entonces $g=h$.

\begin{center}
  \begin{tikzcd}
    C \arrow[bend left]{r}{g} \arrow[bend right]{r}{h} & A \arrow[tail]{r}{f} & B
  \end{tikzcd}
\end{center}
    
\begin{definition} 
  Análogamente, un morfismo $f \in Hom(A,B)$ es \textbf{epimorfismo}
  si para cualquier objeto $C$ y para cada par de morfismos
  $g,h: B \rightarrow C$, el morfismo puede \textit{cancelarse} a la
  derecha:
  
  \begin{gather*}
    g \circ f = h \circ f \quad \Rightarrow \quad g = h
  \end{gather*}
\end{definition}

De nuevo, si el siguiente diagrama conmuta y $f$ es epimorfismo, entonces $g=h$.

\begin{center}
  \begin{tikzcd}
    A \arrow[two heads]{r}{f} & B \arrow[bend left]{r}{g} \arrow[bend right]{r}{h} & C
  \end{tikzcd}
\end{center}

Notaremos que esto coincide con las nociones de función inyectiva y
función sobreyectiva entre conjuntos, pero habrá categorías en las que
las dos nociones no coincidan.

\subsection{Relación con isomorfismos}
Podemos demostrar desde la definición que un isomorfismo deberá ser
forzosamente un monomorfismo y un epimorfismo. Sin embargo, no será
cierto que un morfismo que sea monomorfismo y epimorfismo deba ser
isomorfismo. Ocurre de hecho, en la categoría de los conjuntos,
pero no así en $(\mathbb{Z},\leq)$, por ejemplo:


\section {Ejemplos en varias categorías}
\subsection{Categoría \texttt{Set}}
En la categoría de los conjuntos, los isomorfismos son las funciones
con inversa; lo que coincide con la definición de isomorfismos entre
conjuntos. Además, sabemos que dos conjuntos son isomorfos si y sólo
si tienen la misma cardinalidad.

Los monomorfismos se corresponden con las funciones
inyectivas y los epimorfismos con las funciones sobreyectivas.

\subsection{Categoría de espacios topológicos}
Nótese que ser isomorfismo implica tener una inversa dentro de la
categoría. Puesto que los morfismos en la categoría de los espacios
topológicos son las funciones continuas; los isomorfismos serán las
funciones continuas con inversa continua.

\subsection{Categoría \texttt{$(\mathbb{Z},\leq)$}}
En esta categoría, los únicos isomorfismos son las identidades. Nótese
además que todos los morfismos son monomorfismos y epimorfismos, y,
aun así, no son isomorfismos. Esto sirve de nuevo para no aceptar
directamente la intuición guiada por la clase \texttt{Set}, donde
morfismos que sean monomorfismos y epimorfismos deben ser
isomorfismos. La composición se define como:

\begin{align*}
  (a,b) \circ (b,c) = (a,c)
\end{align*}

Probar que es categoría se reduce a notar que la composición de
morfismos es morfismo (por ser la relación transitiva), que la
composición es asociativa y que existe el morfismo identidad $(a,a)$,
por ser la relación reflexiva.
     
\subsection{Categoría \texttt{$(S,\leq)$}}
Cualquier conjunto parcialmente ordenado tiene una categoría asociada. Los objetos
son sus elementos y los morfismos son casos particulares de la relación de orden.

\begin{gather*}
  Obj((S,\leq)) = S
\end{gather*}

Hay un morfismo entre $a$ y $b$ si y sólo si $a \leq b$:

\begin{align*}
  Hom(a,b)= \twopartdef{(a,b)}{a \leq b}{\emptyset}{\mbox{no}}
\end{align*}

La composición se define como anteriormente: $(a,b) \circ (b,c) = (a,c)$, y es
trivial volver a probar que se trata de una categoría.
\medskip

Un posible diagrama conmutativo de la categoría \texttt{$(\mathbb{N},\leq)$} sería:

\begin{center}
  \begin{tikzpicture}[descr/.style={fill=white,inner sep=2.5pt}]
    \matrix (m) [matrix of math nodes, row sep=3em, column sep=3em]
    { 1 & 3 & 5 \\ & 6 & 7 \\ };
    \path[->,font=\scriptsize]
    (m-1-1) edge node[auto] {$ (1,3) $} (m-1-2)
    (m-1-2) edge node[auto] {$ (3,6) $} (m-2-2)
    (m-2-2) edge node[auto] {$ (6,7) $} (m-2-3)
    (m-1-1) edge node[auto,swap] {$ (1,6) $} (m-2-2)
    (m-1-2) edge node[auto] {$ (3,5) $} (m-1-3)
    (m-1-3) edge node[auto] {$ (5,7) $} (m-2-3);
  \end{tikzpicture}
\end{center}
 

\chapter{Propiedades universales}
Varias construcciones en las diversas estructuras que motivaron las
categorías pueden parecer escogidas arbitrariamente. Sin embargo, la
definición de las propiedades universales las dejará como las únicas
cumpliendo una construcción formal no arbitraria. Además, nos
permitirá descubrir relaciones más profundas entre las construcciones
en las distintas categorías.

\section {Objetos terminales}
\begin{definition}
  El objeto $I \in Obj(\C)$ se dice \textbf{inicial} cuando a cualquier otro objeto llega
  exactamente un morfismo desde él. Es decir:
  \begin{gather*}
    \forall A \in Obj(\C):\quad \#(Hom(I,A)) = 1
  \end{gather*}
\end{definition}

\begin{definition}
  Análogamente, el objeto $F \in Obj(\C)$ se dice \textbf{final}
  cuando desde cualquier otro objeto llega un único morfismo hacia
  él. Es decir:
  \begin{gather*}
    \forall A \in Obj(\C):\quad \#(Hom(A,F)) = 1
  \end{gather*}
\end{definition}
   
Se llama \textbf{objeto terminal} a un objeto inicial o final y
\textbf{objeto cero} a un objeto terminal y final. Una categoría no
tiene por qué tener objetos terminales, y estos no tienen por qué ser
únicos. Pero serán \textit{esencialmente únicos}, es decir, si dos
objetos son ambos iniciales o ambos finales en una categoría, serán
isomorfos, como demostramos a continuación.
    
\begin{theorem}
Los objetos iniciales y los objetos finales de una
categoría son esencialmente únicos:

  \begin{itemize}
  \item Si $I_1,I_2$ son iniciales, $I_1 \cong I_2$
  \item Si $F_1,F_2$ son finales, $F_1 \cong F_2$
  \end{itemize}
\end{theorem}

\begin{proof}
  Para $I_1,I_2$ iniciales, debe cumplirse que existe un sólo
  $f \in Hom(I_1,I_2)$ y que existe un sólo $g \in Hom(I_2,I_1)$. Como
  en $Hom(I_1,I_1)$ y en $Hom(I_2,I_2)$ sólo existe un morfismo, que
  debe ser la identidad, tenemos:
  
  \begin{gather*}
    g \circ f = 1_{I_1} \quad f \circ g = 1_{I_2}
  \end{gather*}
  
  Por lo que son isomorfismos y $I_1 \cong I_2$. Análogamente se prueba para objetos finales.
\end{proof}
    
\section {Ejemplos}

\subsection {Objetos terminales en \texttt{Set}}
En la categoría $\mathtt{Set}$, es objeto inicial el conjunto vacío:
para cualquier otro conjunto $A$, $Hom(\emptyset, A)$ consta sólo de
la inclusión (la función vacía), que podemos notar por $(\emptyset)$.
\\
Todos los conjuntos de un sólo elemento son finales. Sólo hay un
morfismo $c_\ast \in Hom(A,\{\ast\})$, la función constantemente
$\ast$. Obsérvese que entre ellos, todos los conjuntos de un elemento
son isomorfos como afirma el teorema anterior.
    
\subsection {Objetos terminales en \texttt{Grp}}
Entre los grupos, un homomorfismo debe llevar siempre la identidad
hacia la identidad del otro grupo, y además, no existe el grupo
vacío. Esto hace que el objeto terminal y final de la categoría sea el
grupo que tiene como único elemento la identidad, $\{e\}$.  Hacia
cualquier otro grupo existe un único homomorfismo (la imagen de la
identidad debe ser la identidad), y desde cualquier otro grupo existe
un único homomorfismo (la función constantemente $e$).

\subsection {Objetos terminales en $(\mathbb{N},\leq)$}
En esta categoría, $0$ es trivialmente objeto inicial y no hay objetos
finales.  Nótese además que en $(\mathbb{Z},\leq)$ no habría ningún
objeto terminal. En general, en una categoría $(S,\leq)$, los objetos
terminales son el máximo y el mínimo, que no tienen por qué existir.
    
\subsection {Proyección al cociente}
Dado un conjunto $A$, con una relación de equivalencia $\sim$
definimos una categoría que tenga por objetos las funciones (!) que
preserven la relación de equivalencia.  Es decir, las funciones de la
forma:

\begin{center}
  \begin{tikzcd}%[scale=0.5]
    Z \\
    A \arrow{u}{f}
  \end{tikzcd}
\end{center}

Para $Z$ conjunto y $f \in Hom(A,Z)$, cumpliendo
$[a]_{\sim} = [b]_{\sim} \Rightarrow f(a) = f(b)$.

Los morfismos entre dos objetos $Hom((Z,f), (W,g))$ de esta categoría
serán los diagramas conmutativos de la siguiente forma.

\begin{center}
  \begin{tikzcd}
    Z \arrow{r}{\phi} & W \\
    A \arrow{u}{f} \arrow{ur}[swap]{g}
  \end{tikzcd}
\end{center}

Donde, para que se cumpla el diagrama, debería tenerse:
$\phi \circ f = g$.

Y la composición entre morfismos es el exterior de la composición de
los dos diagramas conmutativos, que vuelve a ser diagrama conmutativo:

\begin{center}
  \begin{tabular}{ccccc}    
    $\left( \begin{tikzcd}
      & W \arrow{d}{\psi} \\
      A \arrow{ur}{g} \arrow{r}[swap]{h}
      & V
    \end{tikzcd} \right)$
    & $\circ$ &
    $\left(
      \begin{tikzcd}
          Z \arrow{r}{\phi} & W \\
          A \arrow{u}{f} \arrow{ur}[swap]{g}
        \end{tikzcd}
    \right)$
      &
        $=$
      &
        $\left(
        \begin{tikzcd}
          Z \arrow[bend left]{dr}{\psi \circ \phi} \\
          A \arrow{u}{f} \arrow{r}[swap]{h} & V
        \end{tikzcd}
                                              \right)$
                                                                                 
  \end{tabular}
\end{center}
      
La construcción de esta categoría puede parecer sorprendente o
artificial, pero pronto usaremos construcciones similares para obtener
propiedades básicas.  Por ahora, basta notar que en esta categoría, la
proyección al cociente $(A/_\sim,\sim)$ es el objeto inicial:

\begin{center}
  \begin{tikzcd}
    A/_\sim \arrow{r}{\exists! \tilde f} & Z \\
    A \arrow{u}{\pi_\sim} \arrow{ur}[swap]{f}
  \end{tikzcd}
\end{center}

\begin{proof}
  Definimos $\tilde f ([x]_\sim) = f(x)$. Comprobamos que está bien
  definida porque exigíamos a las funciones que cumplieran
  $[a]_{\sim} = [b]_{\sim} \Rightarrow f(a) = f(b)$.
  
  El diagrama es conmutativo trivialmente: $\tilde f \circ \pi = f$
\end{proof} 
 

\section {Productos y coproductos}
\subsection{Producto}
El producto de dos objetos en una categoría se definirá como otro
objeto tal que cualquier pareja de morfismos que vaya desde un tercer
objeto hacia los dos objetos pueda descomponerse de manera única a
través de su objeto producto.  En términos de diagramas conmutativos,
para $A,B \in Obj(\C)$, su objeto producto, $A \times B$ cumple que,
para cualquier $Z \in Obj(\C)$ que se descomponga así, existe un
único morfismo $\phi$ que lo lleva al producto:

\begin{center}
  \productCD{ 
    A,B,Z,A \times B,
    \pi_A,\pi_B,f,g, \exists! \phi
  }
\end{center}

A ese morfismo $\phi$ que es único lo llamaremos $f \times g$ cuando
sea necesario notarlo de alguna forma.
    
\begin{example} 
  En la categoría \texttt{Set}, el producto de dos conjuntos es el
  producto cartesiano:
  $A \times B = \{(a,b) \; | \; a \in A, \; b \in B \}$ con las
  proyecciones a cada factor. Si tenemos
  otro conjunto $Z$ con dos funciones $f,g$ hacia $A$ y hacia $B$,
  existe una única función hacia $A \times B$ que hace el diagrama
  conmutativo, la función:
  
  \begin{gather*}
    (f \times g)(z) = (f(z),g(z)) \quad \forall z \in Z
  \end{gather*}
  
  Aquí, las funciones $\pi_A$, $\pi_B$, son las proyecciones usuales
  desde el producto. Es decir, el siguiente diagrama es conmutativo:
  
  \begin{center}
    \productCD { 
      A,B,Z,A \times B,
      \pi_A,\pi_B,f,g,f \times g
    }
  \end{center}
\end{example}

\begin{exercise} 
  Demostrar que $f \times g$ es efectivamente la única función que
  hace que el diagrama conmute.
\end{exercise}

    
Esta primera definición puede ser reescrita de forma que además se
demuestre automáticamente la unicidad esencial del objeto producto en
el caso de existir. Recurrimos para ello a definir una categoría
parecida a la que usamos para definir la proyección al cociente.


\subsection{Categorías \texorpdfstring{$\C_{A,B}$}{C A,B}}
Dados objetos $A,B \in Obj(\C)$ para una categoría $\C$. Definimos la categoría $\C_{A,B}$
como la categoría teniendo por objetos los diagramas de la forma:

\begin{center}
  \begin{tikzcd}[column sep=tiny]
    & Z \arrow{dl}[swap]{f_A} \arrow{dr}{f_B} & \\
    A & & B
  \end{tikzcd}
\end{center}

Donde $Z$ es otro objeto con dos morfismos $f_A,f_B$ hacia $A$ y $B$.

Y teniendo por morfismos las funciones $\phi$ en diagramas conmutativos de la siguiente forma:
\begin{center}
  \begin{tikzcd}[column sep=tiny]
    & V \arrow[bend right]{ddl}[swap]{g_A} \arrow[bend left]{ddr}{g_B} \arrow{d}{\phi} & \\
    & Z \arrow{dl}[swap]{f_A} \arrow{dr}{f_B} & \\
    A & & B
  \end{tikzcd}
\end{center}

Que se componen como podría esperarse:
\begin{center}
  \begin{tabular}{ccccc}	
	  $\left(
	    \begin{tikzcd}[column sep=tiny]
	      & V \arrow[bend right]{ddl}[swap]{g_A} \arrow[bend left]{ddr}{g_B} \arrow{d}{\psi} & \\
	      & W \arrow{dl}[swap]{h_A} \arrow{dr}{h_B} & \\
	      A & & B
	    \end{tikzcd}
	  \right)$
	  &
	  $\circ$
	  &
	  $\left(
	    \begin{tikzcd}[column sep=tiny]
	      & Z \arrow[bend right]{ddl}[swap]{f_A} \arrow[bend left]{ddr}{f_B} \arrow{d}{\phi} & \\
	      & V \arrow{dl}[swap]{g_A} \arrow{dr}{g_B} & \\
	      A & & B
	    \end{tikzcd}
	  \right)$
	  &
	  $=$
	  &
	  $\left(
	    \begin{tikzcd}[column sep=tiny]
	      & Z
              \arrow[bend right]{ddl}[swap]{f_A}
              \arrow[bend left]{ddr}{f_B}
              \arrow{d}[description]{\psi \circ \phi} & \\
	      & W \arrow{dl}[swap]{h_A} \arrow{dr}{h_B} & \\
	      A & & B
	    \end{tikzcd}
	  \right)$	  
  \end{tabular}
\end{center}
    
Pues bien, se define el producto como el objeto final de esta categoría.

\begin{exercise} 
  Comprobar que coincide con la definición anterior y observar que esto
  prueba su unicidad esencial.
\end{exercise}
    
\subsection{Coproducto}
Análogamente, el coproducto se define como el objeto final de la
categoría $\C^{A,B}$, que se forma intuitivamente invirtiendo la
dirección de las flechas de la categoría $\C_{A,B}$.  Es decir, tendrá
objetos de la forma:

\begin{center}
  \begin{tikzcd}[column sep=tiny]
    & Z & \\
    A \arrow{ur}{f_A} & & B \arrow{ul}[swap]{f_B}
  \end{tikzcd}
\end{center}

Y los morfismos y su composición se definirán análogamente.
El coproducto de $A$ y $B$, será $A \amalg B$ con dos morfismos $i_A,i_B$ cumpliendo
el diagrama conmutativo siguiente:

\begin{center}
  \coproductCD{
    A,B,Z,A \amalg B,
    i_A,i_B,f,g, {(f,g)}     
  }
\end{center}

\begin{example} 
  En la categoría \texttt{Set}, el coproducto es la unión disjunta de
  dos conjuntos:
  $A \amalg B \cong \{(a,1) \; | \; a \in A\} \cup \{(b,2) \; | \; b
  \in B \}$. Si tenemos otro conjunto $Z$ con dos funciones $f,g$
  desde $A$ y desde $B$, existe una única función desde $A \amalg B$
  que hace el diagrama conmutativo, la función:
  
  \begin{gather*}
    (f,g)((x,n)) = \twopartdef{f(x)}{n = 1}{g(x)}{n = 2}
  \end{gather*}
\end{example}


    
\subsection{Casos particulares}
\subsubsection{Categoría \texttt{Grp}}
En la categoría de los grupos, el producto será el producto usual conjuntista con el
producto de elementos definido componente a componente.
\begin{align*}
  G_1 \times G_2 = \{(a,b)\; |\; a \in A,\; b \in B\} \\
  (a,b) \ast (a',b') = (a \ast a', b \ast b')
\end{align*}

Y el coproducto será el producto libre de grupos, $G \ast H$, formado por las palabras
reducidas de elementos de ambos grupos. El producto de palabras es su yuxtaposición reducida.
\begin{align*}
  G_1 \ast G_2 = \{a_1b_1a_2b_2\dots a_kb_k\; |\; a_i \in A,\; b_i \in B\} \\
  (a_1b_1a_2b_2\dots a_kb_k) \ast (a'_1b'_1a'_2b'_2\dots a'_kb'_k) = (a_1b_1\dots a_kb_ka'_1b'_1\dots a'_kb'_k)
\end{align*}

\subsubsection{Categoría \texttt{($S,\leq$)}}
En la categoría dada por una relación de orden puede demostarse que serán el producto
y el coproducto el ínfimo y el supremo, respectivamente.
\begin{center}
  \begin{tabular}{cc}
    \productCD { 
    a,b,c,{\inf\{a,b\}},
    , , , ,
    }
    &
      \coproductCD{
      a,b,c,{\sup\{a,b\}},
      , , , ,     
      }
  \end{tabular}
\end{center}

\subsubsection{Categoría \texttt{($\mathcal{P}(\Omega),\subseteq$)}}
Esta categoría es similar a la anterior, es un caso particular para subconjuntos
con la inclusión como relación de orden parcial. Un morfismo de $A$ hacia $B$ indica
que $B \subseteq A$. La intersección y la unión serán
aquí el producto y coproducto:
\begin{center}
  \begin{tabular}{cc}
    \productCD { 
    A,B,C,{A \cap B},
    , , , ,
    }
    &
      \coproductCD{
      A,B,C,{A \cup B},
      , , , ,     
      }
  \end{tabular}
\end{center}

\subsubsection{Categoría \texttt{Prop}}
Sea una categoría que tiene por objetos a las proposiciones lógicas. Existe un morfismo
desde una proposición a otra si la primera implica la otra, $A \Rightarrow B$.
En esta categoría, el producto y el coproducto son la conjunción y la disyunción lógicas:
\begin{center}
  \begin{tabular}{cc}
    \productCD { 
    A,B,Z,A \wedge B,
    , , , ,
    }
    &
    \coproductCD{
      A,B,Z,A \vee B,
      , , , ,     
      }
  \end{tabular}
\end{center}
Aquí el producto y coproducto de morfismos (implicaciones) sería de la forma:
\begin{align*}
  (Z \Rightarrow A) \times (Z \Rightarrow B) & = (Z \Rightarrow A \wedge B) \\
  ((A \Rightarrow Z), (B \Rightarrow Z)) & = (A \vee B) \Rightarrow Z
\end{align*}
Nótese que en esta categoría, el objeto inicial es \texttt{False} y el objeto
inicial es \texttt{True}.
    
\subsubsection{Categoría \texttt{Hask}}
Sea ahora la categoría de tipos de Haskell.\footnote{Aunque, hablando estrictamente, 
  no sea ni siquiera una categoría, es interesante trabajar un poco en ella: \url{http://www.haskell.org/haskellwiki/Hask}} Los objetos son los posibles tipos
en Haskell y un morfismo entre dos objetos es una función que tome uno de los
tipos y devuelva el otro.
Tenemos los diagramas siguientes de producto y coproducto:
\begin{center}
  \begin{tabular}{cc}
    \productCD {
    \mathtt{a},\mathtt{b},\mathtt{c},{\quad \mathtt{(a,b)} \quad},
    \mathtt{fst}, \mathtt{snd}, \mathtt{f}, \mathtt{g}, \mathtt{f><g}
    }
    &
      \coproductCD{
      a,b,c,\mathtt{Either\ a\ b},
      \mathtt{Left}, \mathtt{Right}, \mathtt{f}, \mathtt{g}, \mathtt{either\ f\ g}    
      }
  \end{tabular}
\end{center}
Hay ciertas similitudes entre las categorías de tipos y proposiciones que
se quedarán fuera de lo que se expone aquí.

\section{Kernels y cokernels}
Los kernels y cokernels se definirán como sigue:
\begin{center}
  \kernelCD{
    X, Y, K,
    f, k, 0
  }
\end{center}
\begin{center}
  \cokernelCD{
    X, Y, Q,
    f, q, 0
  }
\end{center}

    
\chapter {Functores}
\section{La categoría opuesta}
Hasta ahora, pudiera parecer que estamos repitiendo trabajo. Todo lo
hemos definido dos veces, una vez sobre cada dirección. Composición
a izquierda y derecha, monomorfismos y epimorfismos, objetos
iniciales y finales, productos y coproductos.

Vamos a definir ahora la categoría $\C^{op}$ como la categoría
opuesta de $\C$. Intuitivamente, es la categoría que se obtiene
invirtiendo el sentido de todos los morfismos. Formalmente, esta
categoría está formada por:

\begin{itemize}
\item $Obj(\C^{op}) = Obj(\C)$
\item $Hom_{\C^{op}}(A,B) = Hom_{\C}(B,A)$, para cualesquiera
  $A,B \in Obj(\C^{op})$
\end{itemize}

La composición de morfismos es aquí trivial si tenemos en cuenta
que hemos dado la vuelta a los morfismos:

\begin{gather*}
  \forall f^{op} \in Hom(B,A), g^{op} \in Hom(C,B):\quad f^{op} \circ g^{op} = (g \circ f)^{op}
\end{gather*}

Como podemos observar en estos dos diagramas:

\begin{center}
  \begin{tabular}{cc}
    \begin{tikzcd}
      A \arrow{r}{f} \arrow[bend right]{rr}[swap]{g\circ f} &
      B \arrow{r}{g} &
      C
    \end{tikzcd}
    &
      \begin{tikzcd}
        A  &
        B \arrow{l}[swap]{f^{op}} &
        C \arrow{l}[swap]{g^{op}} \arrow[bend left]{ll}{f^{op} \circ g^{op}}
      \end{tikzcd}
  \end{tabular}
\end{center}

\begin{exercise} 
  Comprobar que la categoría opuesta forma una categoría.
\end{exercise}
\begin{exercise} 
  Definir el objeto final y el coproducto en términos del inicial y
  el producto respectivamente, usando la categoría opuesta.
\end{exercise}
\begin{exercise} 
  Probar que un monomorfismo en una categoría es epimorfismo en la
  opuesta y que los isomorfismos se mantienen también en la categoría
  opuesta.
\end{exercise}

 
\section {Functores}
Una aplicación entre categorías preservando su estructura es un functor.

\begin{definition} Sean $\mathcal{A}, \mathcal{B}$ categorías. Un \textbf{functor} 
  $F: \mathcal{A} \rightarrow \mathcal{B}$ está formado por:
  \begin{itemize}
  \item Una función entre objetos $F: Obj(\mathcal{A}) \rightarrow Obj(\mathcal{B})$
  \item Una función entre morfismos $F: Hom_\mathcal{A}(A,A') \rightarrow Hom_\mathcal{B}(F(A),F(A'))$
  \end{itemize}
  Cumpliendo:
  \begin{itemize}
  \item Respeta la composición: $F(f' \circ f) = F(f') \circ F(f)$
  \item Respeta la identidad: $F(1_A) = 1_{F(A)}$
  \end{itemize}
\end{definition}

Es decir, el functor lleva objetos de una categoría a objetos de otra y los morfismos
entre objetos de una categoría a morfismos de entre las imágenes de los objetos en la
otra categoría.

\subsection {Ejemplos}
\subsubsection{Cardinalidad}
La cardinalidad, que relaciona el orden parcial de los conjuntos finitos con el
orden total entre los naturales, puede verse como un functor:
\begin{align*}
  \#&: \mathtt{(FiniteSet,\subseteq)} \rightarrow \mathtt{(\mathbb{N},\leq)} \\
  \#&(A) = (cardinalidad\ de\ A) \\
  \#&(A \subseteq B) = \#(A) \leq \#(B) 
\end{align*}

\subsubsection{Endofunctores}
Un endofunctor es aquel que va de una categoría hacia sí misma. Un ejemplo
puede ser una función lineal con determinante positivo en $\mathbb{Z}$:

\begin{align*}
  (\lambda)&: \mathtt{(\mathbb{Z},\leq)} \rightarrow \mathtt{(\mathbb{Z},\leq)} \\
  (\lambda)&(n) = \lambda n \\
  (\lambda)&(p \leq q) = (\lambda p \leq \lambda q)
\end{align*}
  
\subsection {Functores de olvido}
Un ejemplo de functores son los informalmente llamados functores de
olvido, que \textit{olvidan} propiedades de una categoría al llevarla
a otra mediante una inclusión.

\begin{example}
  El functor $U: \mathtt{Grp} \rightarrow \mathtt{Set}$ que olvida la
  estructura de grupo. $U(G)$ es el conjunto de elementos del grupo
  $G$ y $U(\phi)$ es la aplicación $\phi$ entre conjuntos, sin verla
  como un homomorfismo.
\end{example}

\subsection {Functores libres}
Los functores libres hacen un papel inverso a los functores de olvido.
Informalmente, llevan una categoría hacia otra con más información
dotándola del mínimo contexto posible.

\begin{example}
  El functor $F: \mathtt{Set} \rightarrow \mathtt{Grp}$ lleva cada
  conjunto $X$ al grupo libre generado por los elementos de $X$, es
  decir, al grupo de palabras reducidas formadas con elementos de
  $X$. Las funciones entre conjuntos se convierten trivialmente en
  homomorfismos entre sus grupos libres generados.
\end{example}

\section {Transformaciones naturales}
  \begin{definition} 
    Una transformación natural $\alpha$ entre dos functores $F,G: \C \rightarrow \mathcal{D}$
    viene determinada por:
    \begin{itemize}
    \item Para cada $X \in \C$, un morfismo: \, $\alpha_X : F(X) \rightarrow G(X)$
    \end{itemize}
    Cumpliendo que:
    \begin{itemize}
    \item Para cualquier morfismo $f \in Hom(X,Y)$ se tenga: \, $\alpha_Y \circ Ff = Gf \circ \alpha_X$
    \end{itemize}
  \end{definition}
  Lo que queda representado en el siguiente diagrama conmutativo de naturalidad:
  \begin{center}
  \begin{tikzpicture}[descr/.style={fill=white,inner sep=2.5pt}]
    \matrix (m) [matrix of math nodes, row sep=3em, column sep=3em]
    { FX & GX \\
      FY & GY \\ };
    \path[->,font=\scriptsize]
    (m-1-1) edge node[auto] {$ \alpha_X $} (m-1-2)
    (m-2-1) edge node[auto] {$ \alpha_Y $} (m-2-2)
    (m-1-2) edge node[auto] {$ Gf $} (m-2-2)
    (m-1-1) edge node[auto,swap] {$ Ff $} (m-2-1);
  \end{tikzpicture}
  \end{center}
 

%----------------------------------------------------------------------------------------
%	BIBLIOGRAPHY
%----------------------------------------------------------------------------------------

\chapter*{Bibliografía}
\addcontentsline{toc}{chapter}{\textcolor{ocre}{Bibliography}}
\section*{Libros}
\addcontentsline{toc}{section}{Books}
\printbibliography[heading=bibempty,type=book]
\section*{Artículos}
\addcontentsline{toc}{section}{Articles}
\printbibliography[heading=bibempty,type=article]

%----------------------------------------------------------------------------------------
%	INDEX
%----------------------------------------------------------------------------------------

\cleardoublepage
\phantomsection
\setlength{\columnsep}{0.75cm}
\addcontentsline{toc}{chapter}{\textcolor{ocre}{Index}}
\printindex

%----------------------------------------------------------------------------------------

\end{document}